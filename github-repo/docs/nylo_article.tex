\documentclass[12pt]{article}
\usepackage{listings}
\usepackage{xcolor}
\usepackage{geometry}
\usepackage{hyperref}
\geometry{margin=1in}
\title{Nylo: A Beginner-Friendly Yet Powerful Programming Language}
\author{DASARI SURYA SAI SRI RAM}
\date{\today}

\definecolor{codegray}{rgb}{0.4,0.4,0.4}
\definecolor{codepurple}{rgb}{0.58,0,0.82}
\definecolor{backcolour}{rgb}{0.95,0.95,0.92}

\lstdefinestyle{nylo}{
    backgroundcolor=\color{backcolour},   
    commentstyle=\color{codegray},
    keywordstyle=\color{blue},
    numberstyle=\tiny\color{gray},
    stringstyle=\color{codepurple},
    basicstyle=\ttfamily\footnotesize,
    breaklines=true,                 
    captionpos=b,                    
    keepspaces=true,                 
    numbers=left,                    
    numbersep=5pt,                  
    showspaces=false,                
    showstringspaces=false,
    showtabs=false,                  
    tabsize=2
}

\begin{document}

\maketitle

\section*{Introduction}
Nylo is a modern interpreted programming language that combines the simplicity of Python, the safety and performance of Rust, the UI capabilities of React, and the command-line integration of Bash.

Designed for beginners but powerful enough for developers, Nylo uses a minimalist syntax and high-level abstractions to make programming intuitive, productive, and expressive.

\section*{Core Design Principles}
\begin{itemize}
  \item \textbf{Minimal Syntax:} Simple, readable, natural language–inspired keywords.
  \item \textbf{High-Level Abstractions:} Avoids low-level memory management.
  \item \textbf{Built-in UI Components:} Supports declarative UI design like React.
  \item \textbf{Shell Integration:} Use shell commands with one line.
  \item \textbf{Cross-Platform and Interpreted:} Runs on Windows, Linux, and macOS.
\end{itemize}

\section*{Basic Syntax}
Here is a simple Nylo program to print a message and use a variable:

\begin{lstlisting}[style=nylo, caption=Hello World in Nylo]
name is "Nylo"
show "Hello from " + name
\end{lstlisting}

\section*{Taking Input}
Nylo allows easy user input using the \texttt{input} keyword:

\begin{lstlisting}[style=nylo, caption=User Input Example]
show "What is your name?"
name is input
show "Hello, " + name + "!"
\end{lstlisting}

\section*{Loops and Lists}
You can easily loop through a list:

\begin{lstlisting}[style=nylo, caption=List and Loop]
items is ["apple", "banana", "cherry"]
for each item in items
  show item.upper()
\end{lstlisting}

\section*{Functions and Buttons}
Nylo supports defining functions and UI components like buttons:

\begin{lstlisting}[style=nylo, caption=Function and Button Example]
to greet do
  show "Welcome!"

button "Click me" on click do
  greet
  message is do "echo Hello from shell!"
  show message
\end{lstlisting}

\section*{Arithmetic Operations}
Nylo can also handle basic math operations:

\begin{lstlisting}[style=nylo, caption=Arithmetic in Nylo]
show "Enter two numbers:"
a is input
b is input

a is float(a)
b is float(b)

sum is a + b
show "Sum: " + str(sum)
\end{lstlisting}

\section*{Conclusion}
Nylo is ideal for beginners learning to code and for developers who want a fast, flexible, and scriptable language with powerful features. With its readable syntax, built-in UI and shell capabilities, and memory-safe execution, Nylo is a modern tool for the next generation of programmers.

\end{document}
